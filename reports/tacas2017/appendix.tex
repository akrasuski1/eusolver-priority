\newpage
\renewcommand{\thesection}{\appendixname~\Alph{section}}
\begin{appendices}
\section{Detailed Experimental Results for the Anytime Extension}
\label{section:appendix_experimental_results}
\begin{table}
\centering
\fontsize{9}{11}\selectfont
\begin{tabular*}{\linewidth}{@{\extracolsep{\fill}}lllllllc}\\\hlx{hv}
\multicolumn{1}{c}{\multirow{2}{*}{Benchmark}} & \multicolumn{1}{c}{Size} & \multicolumn{1}{c}{Time} & \multicolumn{1}{c}{DT Size} & \multicolumn{1}{c}{Size} & \multicolumn{1}{c}{Time} & \multicolumn{1}{c}{DT Size} & Min\\
& \multicolumn{1}{c}{of first} & \multicolumn{1}{c}{to first} & \multicolumn{1}{c}{for first} & \multicolumn{1}{c}{of min} & \multicolumn{1}{c}{to min} & \multicolumn{1}{c}{for min} & first?\\\hlx{hv}
icfp\_103\_10 & 33 & 12.1 & 7 & 26 & 427.4 & 5 & $\times$\\
icfp\_104\_10 & 17 & 1.1 & 5 & 17 & 30.9 & 5 & $\times$\\
icfp\_105\_1000 & 15 & 35.9 & 5 & 11 & 134.9 & 3 & $\times$\\
icfp\_105\_100 & 16 & 3.1 & 5 & 12 & 9.1 & 3 & $\times$\\
icfp\_113\_1000 & 11 & 53.6 & 3 & 11 & 53.6 & 3 & $\checkmark$\\\hlx{h}
icfp\_114\_100 & 20 & 241.6 & 5 & 21 & 793.1 & 5 & $\times$\\
icfp\_118\_100 & 31 & 18.1 & 7 & 16 & 135.8 & 3 & $\times$\\
icfp\_118\_10 & 32 & 4.1 & 7 & 15 & 9.1 & 3 & $\times$\\
icfp\_125\_10 & 57 & 8.8 & 13 & 14 & 437.3 & 3 & $\times$\\
icfp\_134\_1000 & 33 & 1441.1 & 7 & 27 & 2437.4 & 5 & $\times$\\\hlx{h}
icfp\_135\_100 & 13 & 54.0 & 3 & 13 & 54.0 & 3 & $\checkmark$\\
icfp\_139\_10 & 10 & 1.3 & 3 & 10 & 1.3 & 3 & $\checkmark$\\
icfp\_14\_1000 & -- & TO & -- & -- & TO & -- & -- \\
icfp\_143\_1000 & 16 & 1412.5 & 3 & 16 & 1412.5 & 3 & $\checkmark$\\
icfp\_144\_1000 & 24 & 1376.8 & 5 & 24 & 1376.8 & 5 & $\checkmark$\\\hlx{h}
icfp\_144\_100 & 36 & 763.5 & 7 & 26 & 2810.3 & 5 & $\times$\\
icfp\_147\_1000 & 12 & 1114.3 & 3 & 12 & 1114.3 & 3 & $\checkmark$\\
icfp\_150\_10 & 51 & 1.7 & 11 & 24 & 9.3 & 5 & $\times$\\
icfp\_21\_1000 & 21 & 512.6 & 5 & 20 & 1099.5 & 5 & $\times$\\
icfp\_25\_1000 & 22 & 151.6 & 5 & 15 & 1571.4 & 3 & $\times$\\\hlx{h}
icfp\_28\_10 & 2 & 0.04 & 0 & 2 & 0.04 & 0 & $\checkmark$\\
icfp\_30\_10 & 14 & 14.4 & 3 & 14 & 14.4 & 3 & $\checkmark$\\
icfp\_32\_10 & 14 & 9.1 & 3 & 14 & 9.1 & 3 & $\checkmark$\\
icfp\_38\_10 & 21 & 5.9 & 5 & 14 & 54.1 & 3 & $\times$\\
icfp\_39\_100 & 12 & 15.8 & 3 & 12 & 15.8 & 3 & $\checkmark$\\\hlx{h}
\end{tabular*}
\vspace*{1mm}
\caption{Results on the ICFP benchmark suite. All times are in seconds.}
\label{table:anytime_results_1}
\end{table}

\begin{table}[!t]
\vspace*{-1ex}
\centering
\fontsize{9}{11}\selectfont
\begin{tabular*}{\linewidth}{@{\extracolsep{\fill}}lllllllc}\\\hlx{hv}
\multicolumn{1}{c}{\multirow{2}{*}{Benchmark}} & \multicolumn{1}{c}{Size} & \multicolumn{1}{c}{Time} & \multicolumn{1}{c}{DT Size} & \multicolumn{1}{c}{Size} & \multicolumn{1}{c}{Time} & \multicolumn{1}{c}{DT Size} & Min\\
& \multicolumn{1}{c}{of first} & \multicolumn{1}{c}{to first} & \multicolumn{1}{c}{for first} & \multicolumn{1}{c}{of min} & \multicolumn{1}{c}{to min} & \multicolumn{1}{c}{for min} & first?\\\hlx{hv}
icfp\_45\_1000.sl & 8 & 0.1 & 3 & 8 & 0.1 & 3 & $\checkmark$ \\
icfp\_45\_10.sl & 8 & 0.0 & 3 & 8 & 0.0 & 3 & $\checkmark$ \\
icfp\_5\_1000.sl & 16 & 0.3 & 5 & 16 & 0.3 & 5 & $\checkmark$ \\
icfp\_51\_10.sl & 10 & 0.0 & 3 & 10 & 0.0 & 3 & $\checkmark$ \\
icfp\_54\_1000.sl & 10 & 0.1 & 3 & 10 & 0.1 & 3 & $\checkmark$ \\\hlx{h}
icfp\_56\_1000.sl & 65 & 6.5 & 15 & 56 & 272.2 & 11 & $\times$ \\
icfp\_64\_10.sl & 19 & 0.2 & 5 & 19 & 0.2 & 5 & $\checkmark$ \\
icfp\_68\_1000.sl & 58 & 6.9 & 15 & 39 & 286.6 & 11 & $\times$ \\
icfp\_69\_10.sl & 10 & 0.1 & 3 & 10 & 0.1 & 3 & $\checkmark$ \\
icfp\_7\_1000.sl & 23 & 3.2 & 7 & 17 & 216.4 & 5 & $\times$ \\\hlx{h}
icfp\_7\_10.sl & 15 & 0.1 & 5 & 15 & 0.1 & 5 & $\checkmark$ \\
icfp\_72\_10.sl & 12 & 0.1 & 3 & 12 & 0.1 & 3 & $\checkmark$ \\
icfp\_73\_10.sl & 16 & 0.0 & 5 & 12 & 9.8 & 3 & $\times$ \\
icfp\_81\_1000.sl & 22 & 0.5 & 5 & 16 & 228.2 & 3 & $\times$ \\
icfp\_82\_100.sl & 16 & 0.7 & 5 & 16 & 0.7 & 5 & $\checkmark$ \\\hlx{h}
icfp\_82\_10.sl & 16 & 1.1 & 5 & 16 & 1.1 & 5 & $\checkmark$ \\
icfp\_87\_10.sl & 17 & 0.1 & 5 & 17 & 0.1 & 5 & $\checkmark$ \\
icfp\_9\_1000.sl & 14 & 3.4 & 3 & 14 & 3.4 & 3 & $\checkmark$ \\
icfp\_93\_1000.sl & 19 & 0.4 & 5 & 19 & 0.4 & 5 & $\checkmark$ \\
icfp\_94\_1000.sl & 16 & 0.2 & 5 & 16 & 0.2 & 5 & $\checkmark$ \\\hlx{h}
icfp\_94\_100.sl & 16 & 0.1 & 5 & 16 & 0.1 & 5 & $\checkmark$ \\
icfp\_95\_100.sl & 21 & 80.3 & 7 & 21 & 80.3 & 7 & $\checkmark$ \\
icfp\_96\_1000.sl & 43 & 6.3 & 11 & 29 & 511.0 & 7 & $\times$ \\
icfp\_96\_10.sl & 48 & 1.4 & 13 & 24 & 8.8 & 7 & $\times$ \\
icfp\_99\_100.sl & 17 & 0.3 & 5 & 16 & 53.2 & 5 & $\times$ \\\hlx{h}
\end{tabular*}
\vspace*{1mm}
\caption{Results on the ICFP benchmark suite (continued from Table~\ref{table:anytime_results_1}) . All times are in seconds.}
\label{table:anytime_results_2}
\end{table}

% \begin{table}[!t]
\centering
\fontsize{9}{11}\selectfont
\begin{tabular*}{\linewidth}{@{\extracolsep{\fill}}cllllll}\\\hlx{hv}
\multirow{2}{*}{\# Args} & \multicolumn{3}{c}{D\&Q Enumeration} & \multicolumn{3}{c}{CVC4}\\\hlx{vc{2-4,5-7}v}
& Size & Time (s) & \# Comp. & Size & Time (s) & \# Comp\\\hlx{h}
2 & 6 & $< 0.1$ & 1 & 6 & $< 0.1$ & 1\\
3 & 16 & 0.16 & 3 & 25 & $< 0.1$ & 4\\
4 & 36 & 0.77 & 7 & 55 & $< 0.1$ & 9\\
5 & 76 & 6.2 & 15 & 96 & $< 0.1$ & 16\\
6 & 156 & 63.2 & 31 & 148 & $< 0.1$ & 25\\
7 & 316 & 546 & 63 & 211 & $< 0.1$ & 36\\
8 & -- & TO & -- & 285 & $< 0.1$ & 49\\
9 & -- & TO & -- & 370 & 0.1 & 64\\
10 & -- & TO & -- & 466 & 0.15 & 81\\\hlx{h}
\end{tabular*}
\caption{Results on the parameterized ``max'' benchmarks.}
\label{table:max_results}
\end{table}


\vspace*{-1ex}
Detailed results of running our solver on the ICFP benchmarks are
presented in in
Tables~\ref{table:anytime_results_1}~and~\ref{table:anytime_results_2}. For
each benchmark,
we report
\begin{inparaenum}[(a)]
\item the size of the first solution discovered, the time at which it
was discovered, and the associated decision tree size;
\item the same details for the minimal solution discovered; and
\item whether the first solution discovered was itself minimal sized
  solution.
\end{inparaenum}

\comment{
Table~\ref{table:max_results} demonstrates how our algorithm stacks up
against the state-of-the-art CVC4 solver, when evaluated with the
``max'' benchmarks, which are described in Section~\ref{sec:evaluation}.
The columns indicate the sizes of solutions generated, the time to arrive
at the solution, as well as the number of comparisons in the solution
obtained by each of the solvers, for different values of the parameter
$n$, indicated in the first column.
}

\section{Description of the Enumeration Procedure}
\label{section:appendix_esolver}
\begin{algorithm}
  \begin{algorithmic}[1]
    \fontsize{8}{10}\selectfont
    \Require Grammar $\Grammar = \tuple { \NonTerminals, \StartSymbol, \Rules }$ and a set of points $\Points$
    \Ensure Expressions $\tuple { \Expr_1, \Expr_2, \ldots }$ s.t. $\forall i < j : \vert \Expr_i \vert \leq \vert \Expr_j
    \vert \wedge \exists \Point \in \Points : \Expr_i[\Point] \neq \Expr_j[\Point]$
    \ForAll {$\NonTerminal \in \NonTerminals$} $\Productions[\NonTerminals] \gets \emptyset$ \EndFor
    \ForAll {$(\NonTerminal, \Expr) \in \Rules$}\label{line:enumerate:level_one_iter}
    \If { $\Expr \in \Theory[\FormalParameters]$ }
    $\Productions[\NonTerminal] \gets \Productions[\NonTerminal] \cup \{ \Expr \}$  \label{line:enumerate:level_one}
    \EndIf
    \EndFor
    \State $ \Size \gets 1 $
    \While { $\True$ }
    \ForAll {$(\NonTerminal, \Expr) \in \Rules$}
    \State $(\NonTerminal_1, \ldots, \NonTerminal_n) \gets \mbox{List of non-terminals occurring in $\Expr$ }$
    \ForAll { $(\Expr_1, \ldots, \Expr_n) \in \Productions[\NonTerminal_1] \times \cdots \times \Productions[\NonTerminal_n]$ }
    \State $\Expr^* \gets \Expr[\Expr_1/\NonTerminal_1,\ldots,\NonTerminal_n/\Expr_n]$
    \If { $\vert \Expr^* \vert \leq \Size \wedge \forall \Expr' \in
      \Productions[\NonTerminal] . \exists \Point \in \Points :
    \Expr'[\Point] \neq \Expr^*[\Point] $ }
    \State $\Productions[\NonTerminal] \gets \Productions[\NonTerminal] \cup \Expr^*$
    \If { $\NonTerminal = \StartSymbol$ } \textbf{yield} $\Expr^*$ \EndIf
    \EndIf
    \EndFor
    \EndFor
    \State $\Size \gets \Size + 1$
    \EndWhile
  \end{algorithmic}
  \caption{Enumerating distinct expressions from a grammar}
  \label{algo:enumerate}
\end{algorithm}

Algorithm~\ref{algo:enumerate} describes the \textsc{enumerate}
algorithm referred to in Section~\ref{sec:enumeration}. We continue to
use the same notation here that was introduced in Section~\ref{sec:enumeration}.
The \textsc{enumerate} procedure maintains a map $\Productions : \NonTerminals \to
\Powerset{\Theory[\FormalParameters]}$ from non-terminals to
expressions they can produce.
The invariant maintained by the procedure is that every pair of
expressions in $\Productions[\NonTerminal]$ is distinct on $\Points$.

The algorithm starts by first accumulating into
$\Productions[\NonTerminal]$ the expressions that can be produced from
$\NonTerminal$ in one step
(lines~\ref{line:enumerate:level_one_iter}-\ref{line:enumerate:level_one}).
Then, for each possible expression size $\Size$, it attempts to
instantiate each production rule in the grammar with expressions already
generated and stored in $\Productions$, to generate new expressions of
size at most $\Size$.
These newly generated expression are checked for distinctness from
already generated ones, and if so, added to
$\Productions[\NonTerminal]$.
The algorithm returns all the expressions produced from the starting
non-terminal $\StartSymbol$.

\section{Algorithm~\ref{algo:main} without line~\ref{line:main:more_terms}}
\label{sec:app:ex}

Consider the conditional expression grammar given in
Figure~\ref{fig:bad_grammar} and the specification $f(x) = x$.
Note that the grammar can generate terms equivalent to $ax + b$ where
$a, b \geq 0$ and predicates $ax + b < 0$.
The solution expression is obviously $f(x) \equiv x$.


Now, in the first round, the algorithm enumerates the first term, i.e.,
$0$.
Say counter-example point returned is $\Point_1 = \{ x \mapsto 1 \}$.
In the second round, the algorithm proposes the term $1$, and the
say counter-example point returned is $\Point_2 = \{ x \mapsto 2 \}$.
In the third round, the algorithm enumerates the terms $1$ and $2$,
which together cover all the previous counter-examples.

Without line~\ref{line:main:more_terms}, any attempts to enumerate
predicates and learn decision trees will fail here as no predicate that
can be generated by the grammar can distinguish between the $\Point_1$
and $\Point_2$.
However, due to line~\ref{line:main:more_terms}, the algorithm will
continue to generate more terms while also generating predicates.
This will lead to the generation of the term $x$, and the algorithm will
succeed.

\begin{figure}[!t]
\vspace*{3ex}
  \begin{alltt}
  \fontsize{9}{10}\selectfont
                      S ::= T | if (C) then T else T
                      T ::= 0 | 1 | 2 | x
                      C ::= T < 0
\end{alltt}
\caption{Grammar for~\ref{sec:app:ex}}
  \label{fig:bad_grammar}
\end{figure}


\end{appendices}
